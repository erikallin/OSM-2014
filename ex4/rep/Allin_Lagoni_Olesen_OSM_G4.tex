%Standard opsætning
\documentclass[a4paper,12pt]{article}
\usepackage[utf8]{inputenc}
\usepackage[T1]{fontenc}
\usepackage{amsfonts}
\usepackage{graphicx}
\usepackage{float}
\usepackage{hyperref}
\usepackage{listings,xcolor}
\usepackage[vlined, ruled, linesnumbered]{algorithm2e}
\usepackage{pdfpages} %Til at indsætte pdf'er

\usepackage{amssymb}
%Danske symboler
\usepackage[danish]{babel}
%Matematik-ting
\usepackage{mathtools}
\usepackage{setspace}
%Halvanden linjeafstand
\onehalfspacing
%Sidehoved
\usepackage{fancyhdr}
\pagestyle{fancy}
%Rydder fancyheads(sidehoved) underlige tekst
\fancyhf{}
\setlength{\headheight}{15pt}
%Sidetal
\cfoot[]{\thepage}
%Centreret sidehoved
\chead{Dennis Bøgelund Olesen, Emil Lagoni, Erik David Allin}

\renewcommand{\thesubsection}{\alph{subsection}}

\title{Styresystemer og multiprogrammering (OSM) - G4}
\author{Dennis Bøgelund Olesen - 060593 - cwb759 \\ Emil Lagoni - 051290 - frs575 \\ Erik David Allin - 171292 - smt504}
\date{10. marts 2013}

\begin{document}
\maketitle %insert the defined title
\thispagestyle{empty}
\setcounter{page}{0}
\newpage
% Nedenstående 2 linjer bruges til indholdsfortegnelse.
%\tableofcontents
%\newpage

% let's begin



\section*{Task 1}
Hvis disse funktioner bliver kaldt fra kernel, vil exceptions blive håndteret
 fra \texttt{kernel/exception.c}, og hvis kaldet kommer fra user-space håndteres
 det fra \texttt{proc/exception.c}.
\subsection*{tlb\_load/store\_exception}
Denne funktion henter en side hvis denne side findes i en \texttt{pagetable}.
 Hvis dette element findes, eller hvis det element man leder efter, findes i
 pagetable, så indsættes siden i \texttt{tlb}'en, hvis der ikke er plads i denne
 vil fjerne det ældste element og indsætte det nye. Dette
 er dog ikke implementeret. \\
 Hvis siden man søger efter i TLB'en ikke er eksisterende, så vil tlb\_probe
 fange dette, da den tjekker TLB'en igennem for sider med specifikke ASID- og
 VPN2 id'er. Hvis tlb\_probe ikke matcher nogen side, så vil den returnere
 -1.

\subsection*{tlb\_modified\_exception}
Denne laver en kernel\_panic, da den kun bliver kaldt når \texttt{dirty bit} er
 0.

\section*{Task 2}
\subsection*{syscall\_memlimit}
Vi har brugt \texttt{vm\_map} som den ser ud i
 \texttt{proc/process.c:process\_start}.
I forbindelse med dette har vi lavet den så den mapper en page ad gangen, og 
herved undgår vi at remappe de samme pages flere gange. For at mappe forskellige
sider kører vi loopet heap\_end - old\_heap\_end / PAGE\_SIZE - 1, gange.
i old heap end har vi sørget for at processer i process\_start gemmer 3 $\cdot$
 PAGESIZE da trådende i forvejen mapper 3 pages.
 Efter siderne er mappet sætter vi heap\_end i processen, til argumentet heap\_end.

\subsection*{malloc \& free}
I denne har vi ikke kunnet få det til at virke, og derfor er koden i et vist 
omfang meget pseudo. 
Ideén var dog at vi fjernede den gamle byte heap[MAX\_HEAP] og satte free\_list
= heap\_start + sizeof(free\_list)
Derefter satte vi free list -> size = integer versionen af pointeren til 
heap start + PAGE SIZE, så den peger på det bagerste.
Når dette er gjort burde den givne kode kunne køre videre.
Til sidst i kaldet, hvor den ender hvis der ikke er mappet nok plads, mapper vi
en page mere og kalder malloc igen.
\\
I free ville der ikke umiddelbart kræve nogle ændring, da denne ikke mapper noget ny plads, og at vi har forsøgt at beholde free list strukturen.
\end{document}
