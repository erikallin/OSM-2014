%Standard opsætning
\documentclass[a4paper,12pt]{article}
\usepackage[utf8]{inputenc}
\usepackage[T1]{fontenc}
\usepackage{amsfonts}
\usepackage{graphicx}
\usepackage{float}
\usepackage{hyperref}
\usepackage{listings,xcolor}
\usepackage[vlined, ruled, linesnumbered]{algorithm2e}
\usepackage{pdfpages} %Til at indsætte pdf'er
\usepackage{listings} %Til pæn C-kode
\lstset{language=C} %Definerer sproget

\definecolor{mygreen}{rgb}{0,0.6,0}
\definecolor{mygray}{rgb}{0.5,0.5,0.5}
\definecolor{mymauve}{rgb}{0.58,0,0.82}

\lstset{ %
  backgroundcolor=\color{white},   % choose the background color; you must add \usepackage{color} or \usepackage{xcolor}
  basicstyle=\footnotesize,        % the size of the fonts that are used for the code
  breakatwhitespace=false,         % sets if automatic breaks should only happen at whitespace
  breaklines=true,                 % sets automatic line breaking
  captionpos=b,                    % sets the caption-position to bottom
  commentstyle=\color{mygreen},    % comment style
  deletekeywords={...},            % if you want to delete keywords from the given language
  escapeinside={\%*}{*)},          % if you want to add LaTeX within your code
  extendedchars=true,              % lets you use non-ASCII characters; for 8-bits encodings only, does not work with UTF-8
  frame=single,                    % adds a frame around the code
  keepspaces=true,                 % keeps spaces in text, useful for keeping indentation of code (possibly needs columns=flexible)
  keywordstyle=\color{blue},       % keyword style
  language=Octave,                 % the language of the code
  morekeywords={*,...},            % if you want to add more keywords to the set
  numbers=left,                    % where to put the line-numbers; possible values are (none, left, right)
  numbersep=5pt,                   % how far the line-numbers are from the code
  numberstyle=\tiny\color{mygray}, % the style that is used for the line-numbers
  rulecolor=\color{black},         % if not set, the frame-color may be changed on line-breaks within not-black text (e.g. comments (green here))
  showspaces=false,                % show spaces everywhere adding particular underscores; it overrides 'showstringspaces'
  showstringspaces=false,          % underline spaces within strings only
  showtabs=false,                  % show tabs within strings adding particular underscores
  stepnumber=1,                    % the step between two line-numbers. If it's 1, each line will be numbered
  stringstyle=\color{mymauve},     % string literal style
  tabsize=2,                       % sets default tabsize to 2 spaces
%  title=\lstname                   % show the filename of files included with \lstinputlisting; also try caption instead of title
}

\usepackage{amssymb}
%Danske symboler
\usepackage[danish]{babel}
%Matematik-ting
\usepackage{mathtools}
\usepackage{setspace}
%Halvanden linjeafstand
\onehalfspacing
%Sidehoved
\usepackage{fancyhdr}
\pagestyle{fancy}
%Rydder fancyheads(sidehoved) underlige tekst
\fancyhf{}
\setlength{\headheight}{15pt}
%Sidetal
\cfoot[]{\thepage} 
%Centreret sidehoved
\chead{Dennis Bøgelund Olesen, Emil Lagoni, Erik David Allin}

\renewcommand{\thesubsection}{\alph{subsection}}

\title{Styresystemer og multiprogrammering (OSM) - G1}
\author{Dennis Bøgelund Olesen - 060593 - cwb759 \\ Emil Lagoni - 051290 - frs575 \\ Erik David Allin - 171292 - smt504}
\date{17. Februar 2013}

\begin{document}
\maketitle %insert the defined title
\thispagestyle{empty}
\setcounter{page}{0}
\newpage
% Nedenstående 2 linjer bruges til indholdsfortegnelse.
%\tableofcontents
%\newpage

% let's begin


\section*{Task 1}
\subsection*{Insert}
Da en \texttt{doubly linked list} består af pointere til \texttt{nodes}, som består af en \texttt{item} og en pointer, skal listens pointere opdateres samt at pointerne i noderne skal opdateres. Da man kun har en pointer i noderne, så skal man finde den næste, eller forrige, vha. \texttt{xor} mellem den nuværende nodes adresse og den forrige, eller næste, nodes adresse.

\subsection*{Extract}
Denne fjerner det første eller sidste element i listen, alt efter om atTail er 1 eller 0. Den fjerner så dette element og opdatere pointeren for næste element i listen. Derefter frigørers det gamle  elements plads.

\subsection*{Search}
Tager en pointer til en boolsk funktion som allerede er defineret og løber listen igennem enten til at den finder en match eller til at der ikke er flere elementer, hvor den så vil returnere -1. 
Vi kan løbe igennem listen, da vi altid kender addressen på det tidligere element, og derfor kan bruge next = ptr $\wedge$ prev, til at finde det næste element. ptr er her pointeren gemt inde i det nuværende element. 

\subsection*{reverse}
Denne bytter rundt på head og tail i strukturen dlist. Dette virker, da vi har:
\\
next = ptr $\wedge$ prev. Så uanset hvilken side man starter fra, vil man finde det næste element, da man ikke i praksis kan se forskel på hvad vej man går igennem listen.

\subsection*{Testing}
For at teste listen har vi lavet filen \nameref{main.c}. Hvad der bliver og testet og hvordan den gør det ses tydeligt af filen. For at make strukturen køres \textbf{make build all} og for at bygge main.c køres \textbf{make}. Af testen ser vi at strukturen fungerer og alle funktionaliteterne gør som forventet.

\section*{Task 2}
\textbf{Filer involveret:} fs/\texttt{\nameref{io.c}}, fs/\texttt{\nameref{io.h}} samt tests/\texttt{\nameref{readwrite.c}}.
\\[5px]
I denne opgave, udnyttede vi os af typen device fra drivers, som tillod os at bruge kernel-kaldene read og write. Vi skulle altså lave en driver pointer. 
\\
Da device-strukturen har et generisk device i sin struktur, kan vi udnytte den GCD vi har lavet. Vi kan nemlig se af GCD, at den har henholdsvis read og write, som gør nøjagtigt det vi ønsker. 
\\
I forbindelse med dette bruger vi kernel assert til at sikre os, at vi peger på et device.
\\
At \texttt{\nameref{io.c}} og \texttt{\nameref{io.h}} ligger i mappen fs, er taget fra buenos roadmap, som har inddelt read og write som systemkald, der relaterer til filsystemer. som beskrevet i buenos roadmap, side 44. 
\\
\subsection*{Testing}
For at teste readwrite, lavede vi filen \nameref{dlList.c} i mappen tests. 
\\
Efter at have compilet denne, lavede vi:
\\
\texttt{util/tfstool create fyams.harddisk 2048 disk1}
\\
og
\\
\texttt{util/tfstool write fyams.harddisk tests/readwrite readwrite}
\\[5px]
Når dette er lavet. kan testen køres med kommandoen: 
\\
\texttt{fyams-sim buenos 'initprog=[disk1]readwrite'}
\\
Når dette er startet er det muligt at taste i terminalen, hvorefter read så læser det du skriver, og write skriver det ud til terminalen igen. 
\\
I vores test er der brugt en int buffer. Dette betyder, at alt fylder 4 bytes, så der kan altså ikke læses 63 chars, men derimod kun en fjerdedel. 
\\
C kan dog sagtens se chars som integers.

\section*{Bilag}

\section*{dlList.c}
\label{dlList.c}
\begin{lstlisting}
#include <stdlib.h>
#include <stdint.h>
#include "dlList.h"

void insert(dlist *this, item *thing, bool atTail) {
  node *newNode = malloc(sizeof(node));
  newNode->thing = thing;

  if (atTail) {
    newNode->ptr = (this->tail);
    this->tail->ptr = (node*)((uintptr_t)this->tail->
      ptr ^ (uintptr_t)newNode);
    this->tail = newNode;
  }
  else {
    newNode->ptr = (this->head);
    this->head->ptr = (node*)((uintptr_t)this->head->
      ptr ^ (uintptr_t)newNode);
    this->head = newNode;
  }
}


item* search(dlist *this, bool (*matches)(item*)) {
  if (matches(this->head->thing))
    return this->head->thing;

  node *prev = this->head;
  node *next = this->head->ptr;

  while ((node*)((uintptr_t)next->ptr ^ (uintptr_t)prev)) {
    if (matches(next->thing))
      return next->thing;
    node *tmp = next;
    next = (node*)((uintptr_t)next->ptr ^ (uintptr_t)prev);
    prev = tmp;
  }
  if (matches(this->tail->thing))
    return next->thing;

  return 0;
}


void reverse(dlist *this) {
  dlist *tmp = this;
  this->head = tmp->tail;
  this->tail = tmp->head;
}

item* extract(dlist *this, bool atTail) {
  item *ext;
  node *address;
  node *cleanup;
  if (atTail) {
    address = this->tail->ptr;
    ext = this->tail->thing;
    address->ptr = (node*)((uintptr_t)address->
    ptr ^ (uintptr_t) this->tail);
    cleanup = this->tail;
    this->tail = address;
    free(cleanup);
    return ext;
  }
  else {
    address = this->head->ptr;
    ext = this->head->thing;
    address->ptr = (node*)((uintptr_t)address->
    ptr ^ (uintptr_t) this->head);
    cleanup = this->head;
    this->head = address;
    free(cleanup);
    return ext;
  }
}
\end{lstlisting}
\newpage

\section*{dlList.h}\label{dlList.h}
\begin{lstlisting}
#ifndef DL_LIST_H
#define DL_LIST_H

typedef int bool;
typedef void item;

typedef struct node_ {
  item           *thing;
  struct node_   *ptr;
} node;

typedef struct dlist_ {
  node *head, *tail;
} dlist;

/* Inserts an item to either the start or end of the list */
void insert(dlist *this, item* thing, bool atTail);

/* Extracts either the first or last element in the list, 
   remove it from the list and returns the item. */
item* extract(dlist *this, bool atTail);

/* Flips the direction of the links */
void reverse(dlist *this);

item* search(dlist *this, bool (*matches)(item*));
#endif // DL_LIST_H
\end{lstlisting}
\newpage

\section*{main.c}\label{main.c}
\begin{lstlisting}
#include <stdio.h>
#include <stdlib.h>
#include <time.h>

#include "dlList.h"


int main() {

  // Tildeler memory/plads til listen samt dens head og tail.
  node *tail = malloc(sizeof(node));
  node *head = malloc(sizeof(node));
  dlist *liste = malloc(sizeof(dlist));

  /* Tildeler memory/plads til de elementer, 
     der senere bliver indsat via insert. */
  int *i = malloc(sizeof(int));
  int *j = malloc(sizeof(int));
  int *n = malloc(sizeof(int));
  int *k = malloc(sizeof(int));

  // Vaerdier for elementer der senere bliver indsat.
  i = (int*)1;
  j = (int*)2;
  n = (int*)3;
  k = (int*)4;

  // Tail og head tildeles vaerdier.
  tail->thing = j;
  tail->ptr = head;

  head->thing = n;
  head->ptr = tail;

  /* Funktioner for at teste om bestemte vaerdier findes i 
     dllist via search. */
  bool *eqone(int a) {
    return (bool*)(a == 1);
  }

  bool *eqseven(int a) {
    return (bool*)(a == 7);
  }

 // Vaerdier for at tjekke tiden det tager at indsaette elementer.
  clock_t startInsert1, startInsertAll, endInsert1, endInsertAll;


  /* Head og tail tildeles deres pladser i listen.
     Der tjekkes herudover, om de har de korrekte vaerdier 
     via et print. */
  liste->head = head;
  liste->tail = tail;
  printf("Tail: %p\n", liste->tail);
  printf("Head: %p\n", liste->head);


  /* Der saettes en clock for at tjekke programmets 
     hidtige koeretid,
     og der bliver indsat en raekke elementer.
     Senere saettes der to "slut" clocks, der senere 
     bruges til at tjekke tiden det har taget
     at indsaette elementerne. */
  startInsert1 = clock();
  startInsertAll = clock();
  insert(liste, i, 1);
  endInsert1 = clock();
  insert(liste, i, 1);
  insert(liste, i, 1);
  insert(liste, i, 1);
  insert(liste, i, 1);
  insert(liste, i, 1);
  insert(liste, k, 1);
  insert(liste, k, 1);
  insert(liste, k, 0);
  endInsertAll = clock();


  printf("Insertion time for 1 element: %f\n",
    (double)(endInsert1 - startInsert1) / CLOCKS_PER_SEC);

  printf("Insertion time for alle elementer: %f\n",
    (double)(endInsertAll - startInsertAll) / CLOCKS_PER_SEC);

  printf("%p vaerdi af thing i tail\n",tail->thing);
  printf("%p vaerdi af thing i nye tail\n",liste->tail->thing);

  /* Denne test er ikke korrekt. Slet eller fix
  printf("%p pointer i nye tail\n",liste->tail->ptr);
  printf("%p gamle tail (skal vaere lig pointer i nye tail)\n",tail);
  */

  printf("Tester om 1 er i listen. Returner %p, 
    hvilket betyder at den er der.\n",
    search(liste, (item*)eqone));
  printf("Tester om 7 er i listen. Returner %p, 
    hvilket betyder at den ikke er det.\n",
    search(liste, (item*)eqseven));


  // Udkommenter for at teste reverse
  /* Den vil lave print om fra 4,3,2,1 til 1,2,3,4
    reverse(liste);
  */
  printf("\n Foelgende er test for extract\n");
  printf("%p er thing i head (rigtigt hvis = 4)\n",
    liste->head->thing);
  extract(liste,0);
  printf("%p er nu thing i nye head (rigtigt hvis = 3)\n",
    liste->head->thing);
  extract(liste,0);
  printf("%p er nu thing i nye head (rigtigt hvis = 2)\n",
    liste->head->thing);
  extract(liste,0);
  printf("%p er nu thing i nye head (rigtigt hvis = 1)\n",
    liste->head->thing);

  return 0;
}

\end{lstlisting}
\newpage

\section*{io.h}\label{io.h}
\begin{lstlisting}
#ifndef IO_H
#define IO_H

int syscall_read(int fhandle, void *buffer, int length);
int syscall_write(int fhandle, const void *buffer, int length);

#endif // IO_H
\end{lstlisting}
\newpage

\section*{io.c}\label{io.c}
\begin{lstlisting}
#include "drivers/bootargs.h"
#include "drivers/device.h"
#include "drivers/gcd.h"
#include "drivers/metadev.h"
#include "drivers/polltty.h"
#include "drivers/yams.h"
#include "fs/vfs.h"
#include "kernel/assert.h"
#include "kernel/config.h"
#include "kernel/halt.h"
#include "kernel/idle.h"
#include "kernel/interrupt.h"
#include "kernel/kmalloc.h"
#include "kernel/panic.h"
#include "kernel/scheduler.h"
#include "kernel/synch.h"
#include "kernel/thread.h"
#include "lib/debug.h"
#include "lib/libc.h"
#include "net/network.h"
#include "proc/process.h"
#include "vm/vm.h"


int syscall_read(int fhandle, void *buffer, int length) {
  if(fhandle == 0) {
    device_t *dev;
    gcd_t *gcd;

    /* Skaffer device */
    dev = device_get(YAMS_TYPECODE_TTY,0);
    KERNEL_ASSERT(dev != NULL);
    /* skaffer generisk device fra device */
    gcd = (gcd_t *)dev->generic_device;
    KERNEL_ASSERT(gcd != NULL);


    /* Ifoelge drivers/gcd.h, laeser read, "at most len bytes 
       from the device to the buffer. the function returns 
       the number of bytes read." */

    return gcd->read(gcd,buffer,length);
  }
  return -1;

  /*int tmp = fhandle;
  int tmp2 =(int) buffer;
  int tmp3 = length; */
}

int syscall_write(int fhandle, const void *buffer, int length) {
  if(fhandle == 1) {
    device_t *dev;
    gcd_t *gcd;

    /* Skaffer device */
    dev = device_get(YAMS_TYPECODE_TTY,0);
    KERNEL_ASSERT(dev != NULL);
    /* skaffer generisk device fra device */
    gcd = (gcd_t *)dev->generic_device;
    KERNEL_ASSERT(gcd != NULL);


    /* Ifoelge drivers/gcd.h, skriver write, 
       "at most len bytes from the buffer to the device. 
       The function returns the number of bytes read." */
    return gcd->write(gcd,buffer,length);
  }
  return -1;
}
\end{lstlisting}
\newpage

\section*{readwrite.c} \label{readwrite.c}
\begin{lstlisting}
#include "lib.h"
int main(void) {
  int a[100];
  syscall_read(0, a, 100); 
  syscall_write(1, a, 100);
  return 0;
}
\end{lstlisting}


\end{document}
