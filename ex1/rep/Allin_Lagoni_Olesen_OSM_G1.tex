%Standard opsætning
\documentclass[a4paper,12pt]{article}
\usepackage[utf8]{inputenc}
\usepackage[T1]{fontenc}
\usepackage{amsfonts}
\usepackage{graphicx}
\usepackage{float}
\usepackage{hyperref}
\usepackage{listings,xcolor}
\usepackage[vlined, ruled, linesnumbered]{algorithm2e}
\usepackage{pdfpages} %Til at indsætte pdf'er

\usepackage{amssymb}
%Danske symboler
\usepackage[danish]{babel}
%Matematik-ting
\usepackage{mathtools}
\usepackage{setspace}
%Halvanden linjeafstand
\onehalfspacing
%Sidehoved
\usepackage{fancyhdr}
\pagestyle{fancy}
%Rydder fancyheads(sidehoved) underlige tekst
\fancyhf{}
\setlength{\headheight}{15pt}
%Sidetal
\cfoot[]{\thepage} 
%Centreret sidehoved
\chead{Dennis Bøgelund Olesen, Emil Lagoni, Erik David Allin}

\renewcommand{\thesubsection}{\alph{subsection}}

\title{Styresystemer og multiprogrammering (OSM) - G1}
\author{Dennis Bøgelund Olesen - 060593 - cwb759 \\ Emil Lagoni - 051290 - frs575 \\ Erik David Allin - 171292 - smt504}
\date{17. Februar 2013}

\begin{document}
\maketitle %insert the defined title
\thispagestyle{empty}
\setcounter{page}{0}
\newpage
% Nedenstående 2 linjer bruges til indholdsfortegnelse.
%\tableofcontents
%\newpage

% let's begin



\section*{Task 1}

\section*{Task 2}
\textbf{Filer involveret:} fs/\textbf{io.c}, fs/\textbf{io.h} samt tests/\textbf{readwrite.c}.
\\[5px]
I denne opgave, udnyttede vi os af typen device fra drivers, som tillod os at bruge kernel-kaldene read og write. Vi skulle altså lave en driver pointer. 
\\
Da device-strukturen har et generisk device i sin struktur, kan vi udnytte den GCD vi har lavet. Vi kan nemlig se af GCD, at den har henholdsvis read og write, som gør nøjagtigt det vi ønsker. 
\\
I forbindelse med dette bruger vi kernel assert til at sikre os, at vi peger på et device.
\\
At \textbf{io.c} og \textbf{io.h} ligger i mappen fs, er taget fra buenos roadmap, som har inddelt read og write som systemkald, der relaterer til filsystemer. som beskrevet i buenos roadmap, side 44. 
\\
\subsection{Testing}
For at teste readwrite, lavede vi filen readwrite.c i mappen tests0.
\\
Efter at have compilet denne, lavede vi:
\\
\textbf{util/tfstool create fyams.harddisk 2048 disk1}
\\
og
\\
\textbf{util/tfstool write fyams.harddisk tests/readwrite readwrite}
\\[5px]
Når dette er lavet. kan testen køres med kommandoen: 
\\
\textbf{fyams-sim buenos 'initprog=[disk1]readwrite'}
\\
Når dette er startet er det muligt at taste i terminalen, hvorefter read så læser det du skriver, og write skriver det ud til terminalen igen. 
\\
I vores test er der brugt en int buffer. Dette betyder, at alt fylder 4 bytes, så der kan altså ikke læses 63 chars, men derimod kun en fjerdedel. 
\\
C kan dog sagtens se chars som integers.

\end{document}
