%Standard opsætning
\documentclass[a4paper,12pt]{article}
\usepackage[utf8]{inputenc}
\usepackage[T1]{fontenc}
\usepackage{amsfonts}
\usepackage{graphicx}
\usepackage{float}
\usepackage{hyperref}
\usepackage{listings,xcolor}
\usepackage[vlined, ruled, linesnumbered]{algorithm2e}
\usepackage{pdfpages}
%Til at indsætte pdf'er

\definecolor{dkgreen}{rgb}{0,.6,0}
\definecolor{dkblue}{rgb}{0,0,.6}
\definecolor{dkyellow}{cmyk}{0,0,.8,.3}

\hypersetup{
    colorlinks,
    citecolor=cyan,
    filecolor=cyan,
    linkcolor=cyan,
    urlcolor=cyan
}
\usepackage{amssymb}
%Danske symboler
\usepackage[danish]{babel}
%Matematik-ting
\usepackage{mathtools}
\usepackage{setspace}
%Halvanden linjeafstand
\onehalfspacing
%Sidehoved
\usepackage{fancyhdr}
\pagestyle{fancy}
%Rydder fancyheads(sidehoved) underlige tekst
\fancyhf{}
\setlength{\headheight}{15pt}
%Sidetal
\cfoot[]{\thepage} 
%Centreret sidehoved
\chead{Dennis Bøgelund Olesen, Emil Lagoni, Erik David Allin}

\renewcommand{\thesubsection}{\alph{subsection}}

\title{Styresystemer og multiprogrammering (OSM) - G1}
\author{Dennis Bøgelund Olesen - 060593 - cwb759 \\ Emil Lagoni - 051290 - frs575 \\ Erik David Allin - 171292 - smt504}
\date{17. Februar 2013}

\begin{document}
\maketitle %insert the defined title
\thispagestyle{empty}
\setcounter{page}{0}
\newpage
% Nedenstående 2 linjer bruges til indholdsfortegnelse.
%\tableofcontents
%\newpage

% let's begin

\section{Rapport lolz}
\section{Task2}
Filer involveret:
fs/io.c, fs/io.h, \\
test/readwrite.\\

I denne opgave, udnyttede typen device, fra drivers, som tillod os at bruge kernel kaldene read og write. Vi skulle altså lave en driver pointer.
Da device strukturen har et generisk device i sin struktur, og vi kan se af GCD, at den har henholdsvist read og write, som gør nøjagtig det vi ønsker.
Kan vi udnytte den GCD vi har lavet.
I forbindelse med dette bruger vi kernel assert til at sikre os at vi peger på et device.\\
At io.c og io.h ligger i mappen fs, er taget fra buenos roadmap, som har inddelt read og write som systemkald der relatere til filsystemer. som beskrevet på buenos roadmap side 44. 
\\
\subsection{Testing}
For at teste readwrite, lavede vi filen readwrite.c i mappen tests/. \ref{test}
Efter at have compilet denne, lavede vi\\
util/tfstool create fyams.harddisk 2048 disk1\\
util/tfstool write fyams.harddisk tests/readwrite readwrite\\
\\
Når dette er lavet. kan testen køres med:\\
fyams-sim buenos 'initprog=[disk1]readwrite'\\
Når dette er startet kan man taste i termialen og så læser read det du skriver, og write skriver det ud til terminalen igen.\\
I vores test er der brugt en int buffer, dette betyder at alt fylder 4 bytes, så der kan altså ikke læses 63 chars, men derimod kun en fjerde del. \\
C kan dog sagtens anse chars som integers.\\

\end{document}
